\documentclass[10pt,a4paper]{article}
\usepackage{url}
\usepackage{amssymb}
\usepackage{fullpage}
\usepackage{amsmath}

\title{Senior Nanoscience 2011 Assignment 1}
\date{}
\author{D. G. Wilcox \\
		309248035}

\begin{document}

\maketitle

\section*{Question 1}

\begin{itemize}
	\item[(a)]
	\item[(b)]
	\item[(c)]
\end{itemize}

\section*{Question 2}

\begin{itemize}
	\item[(i)] The Ballot Theroem, introduced by Joseph Bertran, is the question "In an election where candiate A recieves $p$ votes and candiate B $q$ votes, with $p > q$, what is the probability that A will be strictly ahead throughout the count?"\footnote{\url{http://en.wikipedia.org/wiki/Bertand's_ballot_theorem}}. The way this explains Brownian motion is by stating that although the amount of particles (and therefore momentum) transfered to a Brownian particle by smaller particles is likely to have a net value of zero of time, for any given instant one direction (in 3D space) will be favoured over the other directions.

	The assumptions made in order for the Ballot theorem to apply are:
		\begin{itemize}
			\item[\textbullet] A candiate from the Ballot Theorem is a direction of motion in Brownian motion.
			\item[\textbullet] That the bombadment of moment from smaller particles is not always equivalent to zero.
			\item[\textbullet] That the quantities involved (ie. mass of Brownian particle, momentum of smaller particles) are such that a significant amount of motion is produced.
		\end{itemize}
	\item[(ii)] Although the question says to provide an alternative explanation to the Ballot Theroem to explain Brownian motion I will instead make the argument that the Ballot Theorem is a viable argument and that it's assumptions are valid.

	Consider the alternative to the claim by the Ballot Theorem. In this scenario we have multidutes of particles hitting the Brownian Particle over time from all directions (in approximately equal quantities).
\end{itemize}

\section*{Question 3}

\begin{enumerate}
	\item 
		\begin{itemize}
			\item[(i)]
			\item[(ii)]
		\end{itemize}
	\item
		\begin{itemize}
			\item[(i)]
			\item[(ii)]
			\item[(iii)]
		\end{itemize}
	\item
	\item
		\begin{itemize}
			\item[(a)]
			\item[(b)]
			\item[(c)]
		\end{itemize}
\end{enumerate}

\end{document}
